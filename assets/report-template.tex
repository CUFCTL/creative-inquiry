\documentclass[paper=a4, fontsize=11pt,twoside]{scrartcl} 

\usepackage[a4paper,pdftex]{geometry}
\setlength{\oddsidemargin}{5mm}
\setlength{\evensidemargin}{5mm}

\usepackage[english]{babel}
\usepackage[protrusion=true,expansion=true]{microtype}	
\usepackage{amsmath,amsfonts,amsthm,amssymb}
\usepackage{graphicx}

% --------------------------------------------------------------------
% Definitions (do not change)
% --------------------------------------------------------------------
\newcommand{\HRule}[1]{\rule{\linewidth}{#1}}

\makeatletter
\def\printtitle{
	{\centering \@title\par}}
\makeatother

\makeatletter
\def\printauthor{
	{\centering \large \@author}}
\makeatother

% --------------------------------------------------------------------
% Metadata
% --------------------------------------------------------------------
\title{
    \normalsize \textsc{Future Computing Technologies Lab: Creative Inquiry} \\ [2.0cm]
	\HRule{0.5pt} \\
	\LARGE \textbf{\uppercase{Semester Report}}
	\HRule{2pt} \\ [0.5cm]
	\normalsize \today
}

\author{
	Your Name\\	
	Clemson University\\	
	Department of Electrical and Computer Engineering\\
	\texttt{your@email.com} \\
}


% --------------------------------------------------------------------
% Main Document
% --------------------------------------------------------------------
\begin{document}

\thispagestyle{empty}

\printtitle
\vfill
\printauthor
\newpage

\setcounter{page}{1}

\section{Introduction}

In this section you will describe the project that you chose. What data did you work with? What task(s) did you try to perform with the data (visualization, classification, clustering, etc.)? What motivated you to choose these things?

\section{Materials and Methods}

In this section you will describe the steps you took in your project. You can be as detailed as you like, including not only what your code does but also how you prepared your data, how you collected results, etc. Someone reading this section should be able to reproduce what you did on a conceptual level. You do not need to include code, but if you do, please include it as an appendix rather than in the text.

\section{Results}

In this section you will put any relevant results that you have obtained during the semester. You should focus on results that show the dataset you used as well as the tasks that you were able to accomplish with it. Be sure to explain all tables and figures that you provide. For example, if you have a confusion matrix for a classifier which shows especially low accuracy in a particular class, try to offer an explanation for why the classifier performed poorly with that class. Try to present your results in tables and figures rather than in a sentence so that they are clearly distinguishable.

\section{Experience}

\textit{This section is especially important.} In this section you will describe your overall experience in the CI. What did you enjoy? How do you think the CI could be improved (please don't hesitate to speak freely)? What new skills did you learn? What challenges did you face, and if applicable, how did you overcome them? What topics were you exposed to that are otherwise inaccessible to undergraduate students? Is there anything that you would like to pursue in the future as a result of your participation in the CI? If the CI has had any impact on your understanding of graduate research or interest in pursuing a graduate degree, feel free to include that here as well.

\section{Conclusions and Future Work}

In this section you can make conclusions about the results that you obtained during the semester. What are the potential "next steps" for your project, should you choose to continue it? Additionally, you can include any ideas for other projects that you would like to work on, or even general suggestions for projects that you think would fit well in the CI.

\section*{TeX Helpful Hints}\label{design}

Below you will find a few helpful hints for creating different objects (i.e. tables, mathematics, etc.) in a TeX document. You should remove this section from your report before submitting it.

\subsection*{Text}

Use \texttt{section}s and \texttt{subsection}s to organize your document. \LaTeX{} handles all the formatting and numbering automatically. Use \texttt{ref} and \texttt{label} for cross-references --- for example, this is Section \ref{design}. You can also create math expressions in sentences such as $\frac{\pi}{2}$.

\subsection*{Tables and Figures}

You can create a table as shown in Table \ref{tab:table1}. You can upload your figures (JPEG, PNG, or EPS) and include them in the document as shown in Figure \ref{fig:your-figure}.

\begin{table}
	\centering
	\caption{Caption for the table.}
	\label{tab:table1}
	\begin{tabular}{l|c||r}
		\hline
		1 & 2 & 3\\
		\hline
		a & b & c\\
		\hline
	\end{tabular}
\end{table}

\begin{figure}[h!]
\includegraphics[width=\textwidth]{image.jpg}
\caption{\label{fig:your-figure}Example Image}
\end{figure}

\subsection*{Mathematics}

\LaTeX{} is great for typesetting mathematics. Let $X_1, X_2, \ldots, X_n$ be a sequence of independent and identically distributed random variables with $\text{E}[X_i] = \mu$ and $\text{Var}[X_i] = \sigma^2 < \infty$, and let

$$S_n = \frac{X_1 + X_2 + \cdots + X_n}{n} = \frac{1}{n}\sum_{i}^{n} X_i$$

denote their mean. Then as $n$ approaches infinity, the random variables $\sqrt{n}(S_n - \mu)$ converge in distribution to a normal $\mathcal{N}(0, \sigma^2)$.

\subsection*{Lists}

You can make lists with automatic numbering \dots

\begin{enumerate}
	\item Like this,
	\item and like this.
\end{enumerate}

\dots or bullet points \dots

\begin{itemize}
	\item Like this,
	\item and like this.
\end{itemize}

\end{document}
